\documentclass[10pt,a4paper]{article}
%\setkomafont{disposition}{\normalfont\bfseries}
\usepackage[utf8]{inputenc}

\title{Methods: Game Theory I -- Problem set 2}
\author{Cleave, Blair u1275847 \\ Gonzalez, Carlos u1272568 \\ Nguyen Huy, Tung u1273269\\ Wipusanawan, Chayanin u1272560}
\date{}
\usepackage[top=1in, bottom=1.25in, left=1.3in, right=1.3in]{geometry}
\usepackage{graphicx}
\usepackage{setspace}
\usepackage{amsmath}
\usepackage{amssymb}
%\onehalfspacing
\setstretch{1.25}

\hyphenation{}
%\numberwithin{equation}{subsection}
\allowdisplaybreaks
\setcounter{page}{1}
\begin{document}
\maketitle
\section*{Problem 1.}
\subsection*{Solution} 
\paragraph*{a)}
Let $\pi_i$ denote the profit (payoff) of company $i$ given its produced quantity ($q_i$) and the total quantity produced by the rest of the companies $q_{-i}$. Using these notations, the profit of company $i \in {1, 2, 3}$:
\begin{equation}
	\pi_i = \left[ 100 - (q_i + q_{-i}) \right] q_i - 20 q_i
\end{equation}
The first order condition (since the coefficient of the quadratic term is negative, this will provide the profit maximizing quantity) and the best response of company $i \in {1,2,3}$ given other companies' output are:
\begin{equation}\
	\label{q123st1}
	{#eq: q123st1}
	\frac{\partial \pi_i}{\partial q_i} = 80 - 2 q_i - q_{-i} = 0; ~~~\Rightarrow~~~ q_i=40-\frac{q_{-i}}{2}
\end{equation}
As for the 4th company, the profit, the first order condition and the best response function are:
\begin{equation}
	\pi_4 = \left[ 100 - (q_4 + q_{-4}) \right] q_4 - (20+\varphi) q_4
\end{equation}
\begin{equation}
	\label{q4st1}
	\frac{\partial \pi_4}{\partial q_4} = 80 - 2 q_4 - q_{-4}-\varphi = 0; ~~~\Rightarrow~~~ q_4=40-\frac{q_{-4}+\varphi}{2}
\end{equation}
In Nash-equilibrium every company produces the best response quantity given the other companies' output, thus denoting with $q_1^*$,$q_2^*$,$q_3^*$ and $q_4^*$ the produced quantities in equilibrium by a given company, substituting company 1's best response to company 2's best response in equilibrium and then rearranging the equation gives us:
\begin{equation}
	\label{q2st2}
	q_2^* = 40 - \frac{\left( 40 - \frac{q_2^*+q_3^*+q_4^*}{2} \right)+q_3^*+q_4^*}{2} ~~~\Rightarrow~~~ q_2^*=\frac{80-q_3^*-q_4^*}{3}
\end{equation}
Plugging this into company 1's best response in equlibrium, we'll get:
\begin{equation}
	\label{q1st2}
	q_1^* = 40 - \frac{\left( \frac{80-q_3^*-q_4^*}{3} \right)+q_3^*+q_4^*}{2} ~~~\Rightarrow~~~ q_1^*=\frac{80-q_3^*-q_4^*}{3}
\end{equation}
Substituting these two formulas in the best reponse of company 3 in equilibrium and rearranging the equation, we get:
\begin{equation}
	\label{q3st3}
	q_3^*=40-\frac{\left( \frac{80-q_3^*-q_4^*}{3} \right)+\left( \frac{80-q_3^*-q_4^*}{3} \right) + q_4^*}{2} ~~~\Rightarrow~~~ q_3^*=20-\frac{q_4^*}{4}
\end{equation}
Again, plugging this back to (\ref{q2st2}) and (\ref{q1st2}), we get:
\begin{equation}
	\label{q12st3}
	q_2^* = q_1^*=20-\frac{q_4^*}{4}
\end{equation}
And finally, substituting all these in company 4's best response in equlibrium, we get:
\begin{equation}
	\widehat{q_4^*}=40-\frac{3 \cdot \left( 20-\frac{q_4^*}{2} \right)+\varphi}{2} ~~~\Rightarrow~~~ \widehat{q_4^*} = 16-\frac{4}{5} \varphi
\end{equation}
Thus in equlibrium:
\begin{equation}
	q_4^* =
	\begin{cases}
		16-\frac45 \varphi & \text{if} ~ \varphi \leq 20 \\
		0 & \text{if} ~ \varphi > 20
	\end{cases}
\end{equation}
\begin{equation}
	q_1^* = q_2^* = q_3^* =
	\begin{cases}
		16+\frac15 \varphi & \text{if} ~ \varphi \leq 20 \\
		20 & \text{if} ~ \varphi > 20
	\end{cases}
\end{equation}
\begin{equation}
	P^* = (100-Q^*)=
	\begin{cases}
		36-\frac15 \varphi & \text{if} ~ \varphi \leq 20 \\
		40 & \text{if} ~ \varphi > 20
	\end{cases}
	~~~
\end{equation}
\begin{equation}
	\pi_1^*=\pi_2^*=\pi_3^* =
	\begin{cases}
		\left( 16+\frac{\varphi}{5} \right)^2 & \text{if} ~ \varphi \leq 20 \\
		400 & \text{if} ~ \varphi > 20
	\end{cases}
\end{equation}
\begin{equation}
	\label{pi4a}
	\pi_4^*=
	\begin{cases}
		\left( 16-\frac{4}{5}\varphi \right)^2 & \text{if} ~ \varphi \leq 20 \\
		0 & \text{if} ~ \varphi > 20
	\end{cases}	
\end{equation}

\paragraph*{b)} Since the 2 merging company have the same constant marginal costs and neither have fixed costs, given a total output of the merged company ($q_{12}$) it does not matter how they split up the production of that output. Thus marginal costs will be inherited and we can handle the remaining market just like if we have 2 companies with $MC=20$ and one company with $MC=(20+\varphi)$. Thus we can still use the best response function formulated in (\ref{q123st1}) @eq:q123st1 and (\ref{q4st1}). Again, in equilibrium every company produces the best response quantity given the other companies' output, thus:
\begin{equation}
	q_{12}^* = 40 - \frac{q_3^*+q_4^*}{2} ~~;~~ q_{3}^* = 40 - \frac{q_{12}^*+q_4^*}{2} ~~;~~ q_{4}^* = 40 - \frac{q_{12}^*+q_3^*+\varphi}{2}
\end{equation}
Plugging $q_{12}^*$ to $q_3^*$ and rearranging it to $q_3^*$, and finally substituting this one back in $q_{12}^*$, we get:
\begin{equation}
	\label{q123-p2_s3}
	q_3^*=q_{12}^*=\frac{80-q_4^*}{3}
\end{equation}
Substituting these to $q_4^*$ and than substituting this one back in (\ref{q123-p2_s3}), we get:
\begin{equation}
	q_4^*=
	\begin{cases}
		20-\frac34\varphi & \text{if}~\frac{80}{3}\geq \varphi \\
		0 & \text{if}~\frac{80}{3}<\varphi
	\end{cases}
	~~~~
	q_{12}^*=q_3^*=
	\begin{cases}
		\frac{\varphi+80}{4} & \text{if}~\frac{80}{3}\geq \varphi \\
		\frac{80}{3} & \text{if}~\frac{80}{3}<\varphi
	\end{cases}
\end{equation}
Thus:
\begin{equation}
	P^*= (100-Q^*) =
	\begin{cases}
		40+\frac14 \varphi & \text{if}~\frac{80}{3}\geq \varphi \\
		\frac{140}3 & \text{if}~\frac{80}{3}<\varphi
	\end{cases}
\end{equation}
\begin{equation}
	\pi_{12}^*=\pi_3^*=
	\begin{cases}
		\left( 20+\frac14\varphi \right) \left( \frac{\varphi+80}{4} \right) = \left( \frac{\varphi+80}{4}\right)^2& \text{if}~\frac{80}{3}\geq \varphi \\
		\frac{80}3 \cdot \frac{80}3 & \text{if}~\frac{80}{3}<\varphi
	\end{cases}
\end{equation}
\begin{equation}
	\label{pi4qb}
	\pi_{4}^*=
	\begin{cases}
		\left( 20+\frac14\varphi \right) \left( 20-\frac34 \varphi \right) & \text{if}~\frac{80}{3}\geq \varphi \\
		0 & \text{if}~\frac{80}{3}<\varphi
	\end{cases}
\end{equation}
\paragraph*{c)} If $\varphi > 0$, then since company 4 has higher constant marginal costs than company 1, in order to maximize profit (minimize cost) the merged company should shift all production to company 1. Thus, in this case we'll have 3 symmetrical companies on the market. Using the formula from Problem set 1 - Problem 6, we get ($i \in {2,3,'14'}$):
\begin{equation}
	q_i^* = \frac{100-20}{3+1}=20  \pi_i^* = \left( 20 \right)^2 = 400
\end{equation}
Whether it is profitable in this case for company 4 to merge with company 1 depends on the relation between the half of this profit (200) - since company 1 and 4 splits the total profit halfway - and company 4's profit from part a) (formula in (\ref{pi4a})). If $\varphi > 20$, the since $200>0$, merging is profitable for company 4. If $\varphi \leq 20$, then it depends on the relation below:
\begin{equation}
200>\left( 16-\frac{4}{5}\varphi \right)^2 ~~\Leftrightarrow~~ \frac{5}{2}(8-5\cdot\sqrt{2}) \approx 2.32 < \varphi <\frac{5}{2}(8+5\cdot\sqrt{2}) \approx 37.67
\end{equation}
So only if $\varphi > 2.32$, then for company 4 it is profitable to merge with company 1.

On the other hand, if $\varphi < 0$, then since in order to maximize profit, the all the output of the merged company should be produced by company 4. In this case, the market will basically have the same structure as describe in part b), but company 4 will get only half the payoff indicated above (\ref{pi4qb}) - again, since company 1 and 4 splits the profit. In this case, since $\varphi < 0$, the profit of company 4 ($\pi_4^*>0$) will always be higher without merging than in case of merging ($0$).

\break
\section*{Problem 3.}
Find all pure and mixed-strategy Nash equilibria of the following game!

\begin{center}
\begin{tabular}{ r c|c|c|c| }
& \multicolumn{1}{r}{} & \multicolumn{3}{c}{Player 2} \\
& \multicolumn{1}{r}{} &  \multicolumn{1}{c}{L} & \multicolumn{1}{c}{M} & \multicolumn{1}{c}{R} \\
\cline{3-5}
& T & 2,2 & 0,3 & 1,3 \\
\cline{3-5}
Player 1 & N & 3,2 & 1,1 & 0,2 \\
\cline{3-5}
\end{tabular}
\end{center}
\subsection*{Solution} 
First of all, we find the pure strategy Nash equilibria using the best responses correspondences. Let $b_1(s_2)$ be the best response function for Player 1 given the strategy of Player 2 vice versa. We can determine the following best responses for both players immediately from the payoff matrix. There are two pure-strategy Nash equilibria in this game: $(B,L)$ and $(T,R)$.

\begin{align*}
b_1(L) &= \{B\} \\ 
b_1(M) &= \{B\} \\
b_1(R) &= \{T\} \\
b_2(T) &= \{M,R\} \\
b_2(B) &= \{L,R\} 
\end{align*}

Now we look for a mixed-strategy Nash equilibrium. A strategy profile $\sigma = (\sigma_1, \sigma_2)$ is a Nash equilibrium if and only if $u_i(s_i,\sigma_{-i}) = u_i(s'_i \sigma_{-i})$ for all $s_i, s'_i \in S^+_i$. Let $\sigma_{is}$ represents the probability that Player $i = 1,2$ assigns to their pure strategy $s_i$. As $\Sigma \sigma_i = 1$, we can define the strategy of Player 1 as $(\sigma_{1T},1-\sigma_{1T}$ and Player 2 as $\sigma_{2L}, \sigma_{2M}, 1-\sigma_{2L}-\sigma_{2M})$. We establish the expected payoff for each pure strategy of Player 1 as folow.

\begin{align*}
u_1 (T, \sigma_{2L}, \sigma_{2M}) &= 2\cdot(\sigma_{2L}) + 0\cdot(\sigma_{2M}) + 1\cdot(1-\sigma_{2L}-\sigma_{2M}) \\
&= \sigma_{2L} - \sigma_{2M} + 1 \\
u_1 (B, \sigma_{2L}, \sigma_{2M}) &= 3\cdot(\sigma_{2L}) + 1\cdot(\sigma_{2M}) + 0\cdot(1-\sigma_{2L}-\sigma_{2M}) \\
&= 3\cdot\sigma_{2L} + \sigma_{2M}
\end{align*}

We can compute the best response function for Player 1, given the expected payoff of each of their pure strategies.

\begin{align*}
u_1(T, \sigma_{2L}, \sigma_{2M}) &\lesseqgtr u_1(B, \sigma_{2L}, \sigma_{2M}) \\
\sigma_{2L} - \sigma_{2M} + 1  &\lesseqgtr 3\cdot\sigma_{2L} + \sigma_{2M} \\
1 - 2 \cdot \sigma_{2L} &\lesseqgtr 2 \cdot \sigma_{2M}
\end{align*}

As $\sigma_{2R} = 1 - \sigma_{2L} - \sigma_{2M}$, we can rearrange the relation in terms of $\sigma_{2R}$. Also, for $\sigma_{2R} = \sigma_{2L} + \sigma_{2M}$ to be true given $\sigma_{2L} + \sigma_{2M} + \sigma_{2R} =1$, we can derive that $\sigma_{2R} = 1/2$. Finally, we arrive at the best response function of Player 1 in term of $\sigma_{2R}$.

\begin{align*}
\sigma_{2R} &\lesseqgtr \sigma_{2L} + \sigma_{2M}
\end{align*}

\[
b_1(\sigma_{2R}) =
\left \{
  \begin{array}{ccc}
  0 & $if$ & \sigma_{2R} < 1/2 \\
  {[0.1]} & $if$ & \sigma_{2R} = 1/2 \\
  1 & $if$ & \sigma_{2R} > 1/2
  \end{array}
\right.
\]

We similarly derive expected payoff for each of Player 2's pure strategies.

\begin{align*}
u_2 (L, \sigma_{1T}) &= 2\cdot(\sigma_{1T}) + 2\cdot(1-\sigma_{1T}) \\
&= 2 \\
u_2 (M, \sigma_{1T}) &= 3\cdot(\sigma_{1T}) + 1\cdot(1-\sigma_{1T}) \\
&= 2\cdot\sigma_{1T} + 1 \\
u_2 (R, \sigma_{1T}) &= 3\cdot\sigma_{1T} + 2\cdot(1-\sigma_{1T}) \\
&= \sigma_{1T} + 2
\end{align*}

When we consider the condition that Player 2 will assign positive probability for all three pure strategies ($\sigma_{2L}, \sigma_{2M}$, and $\sigma_{2R} > 0$), such strategy can be part of a mixed-strategy Nash equilibrium if and only if $u_2(L, \sigma_{1T}) = u_2(M, \sigma_{1T}) = u_2(R, \sigma_{1T})$. There is no such $\sigma_{1T} \in {[0,1]}$ that satisfy this condition. Therefore, there is no Nash equilibrium in which Player 2 assign positive probability to all three pure strategies.

Then, we consider the cases where Player 2 assign positive probability to only two of their pure strategies. Player 2 may mix:

\begin{itemize}
\item pure strategies L and M only if $2 = 2 \cdot \sigma_{1T} + 1 \geq \sigma_{1T} + 2$. There is no such $\sigma_{1T}$ that satisfies the conditions. Therefore, a mix strategy $(\sigma_{2L}, \sigma_{2M}, 0)$ for $\sigma_{2L}, \sigma_{2M} \in {[0,1]}$ does not belong to Player 2's best response function.
\item pure strategies L and R only if $2 = \sigma_{1T} + 2 \geq 2 \cdot \sigma_{1T} + 1$. In this case, $\sigma_{1T} = 0$ satisfies the condition. Therefore,  $(\sigma_{2L}, 0, \sigma_{2R})$ for $\sigma_{2L}, \sigma_{2R} \in {[0,1]}$ is a best response for $\sigma_{1T} = 0$.
\item pure strategies M and R only if $2 \cdot \sigma_{1T} + 1 = \sigma_{1T} + 2 \geq 2.$ In this case, $\sigma_{1T} = 1$ satisfies the condition. Therefore,  $(0, \sigma_{2M}, \sigma_{2R})$ for $\sigma_{2M}, \sigma_{2R} \in {[0,1]}$ is a best response for $\sigma_{1T} = 1$.
\end{itemize}

From the best response functions of the two players, we look for the correspondence, and find the mixed-strategy Nash equilibria as the following continua:

\begin{itemize}
\item $\{(\sigma_{1T}, \sigma_{1B}), \sigma_{2L}, \sigma_{2M}, \sigma_{2R}) | (\sigma_{1T} = 0, \sigma_{2M} = 0, \sigma_{2R} \leq 1/2\}$ and
\item $\{(\sigma_{1T}, \sigma_{1B}), \sigma_{2L}, \sigma_{2M}, \sigma_{2R}) | \sigma_{1T} = 1, \sigma_{2L}=0, \sigma_{2R} \geq 1/2\}$
\end{itemize}

\section*{Problem 4.}
%\paragraph*{Cournot market with 3 companies} <description of the problem>

\subsection*{Solution} 
\paragraph*{a)}
Using backward induction, we first find the optimal actions in the ``last'' subgame, i.e. the actions of firms 2 and 3. The profit functions for Firms 2 and 3, respectively, are as follows:

\begin{align*}
	\pi_2 &= P \cdot q_2 - c \cdot q_2 \\
	&= (a - q_1 - q_2 - q_3) \cdot q_2 - c \cdot q_2 \\
	\pi_3 &= P \cdot q_3 - c \cdot q_3 \\
	&= (a - q_1 - q_2 - q_3) \cdot q_3 - c \cdot q_3 \\
\end{align*}
We find the profit-maximising quantities $q_i*$ for $i=2,3$ at $\frac{\partial \pi_i}{\partial q_i} = 0$. Given that firms 2 and 3 move simultaneously, the analysis is similar to a static game, with $q_1$ as given.
\begin{align*}
	\frac{\partial \pi_2}{\partial q_2} &= a - q_1 - q_2 - 2q_3 - c_2 = 0 \\
 	q_2* &= \frac{1}{2} (a - q_1 - q_3 - c_2)  \\
	\frac{\partial \pi_3}{\partial q_3} &= a - q_1 - q_2 - 2q_3 - c_3 = 0 \\
 	q_3* &= \frac{1}{2} (a- q_1 - q_2 -c_3)  
\end{align*}
We solve the system of two-variable equations (1) and (2).
\begin{align*}
	q_3* &= \frac{1}{2} (a - q_1 - (\frac{1}{2} (a - q_1 - q_3 - c_2) - c_3) \\
	&= \frac{1}{3} (a - q_1 + c_2 - 2c_3) \\
	q_2* &= \frac{1}{2} (a - q_1 - \frac{1}{3} (a - q_1 + c_2 - 2c_3) - c_2) \\
	&= \frac{1}{3} (a = q_1 - 2c_2 + c_3)
\end{align*}

Now we have the optimal actions for firms 2 and 3 given the action of firm 1 in the earlier stage of the game. Then, we  can consider the longer subgame, that is from the start in which firm 1 moves. Given the optimal actions of firms 2 and 3 from (3) and (4), we can establish the payoffs for firm 1.

\begin{align*}
	\pi_1 &= P \cdot q_1 - c \cdot q_1 \\
	&= (a - q_1 - q_2 - q_3) \cdot q_1 - c \cdot q_1 \\
	&= (a - q_1 - \frac{1}{3} (a = q_1 - 2c_2 + c_3) - \frac{1}{3} (a - q_1 + c_2 - 2c_3)) \cdot q_1 - c \cdot q_1 \\
	&= \frac{1}{3}(a - q_1 + c_2 + c_3) \cdot q_1 - c_1 q_1
\end{align*}
Again, we find the profit-maximising quantity $q_1*$ at $\frac{\partial \pi_i}{\partial q_i} = 0$,
\begin{align*}
	\frac{\partial \pi_2}{\partial q_2} &= \frac{1}{3}(a - 2q_1 + c_2 + c_3) - c_1 = 0 \\
	q_1* &= \frac{1}{2} (a - 3c_1 + c_2 + c_3)
\end{align*}
Plugging this back to (3) and (4).
\begin{align*}
	q_2* &= \frac{1}{3} (a - \frac{1}{2} (a - 3c_1 + c_2 + c_3) - 2c_2 + c_3) \\
	&= \frac{1}{6} (a + 3c_1 - 5c_2 + c_3) \\
	q_3* &= \frac{1}{3} (a -\frac{1}{2} (a - 3c_1 + c_2 + c_3) + c_2 - 2c_3) \\
	&= \frac{1}{6} (a + 3c_1 + c_2 - 5c_3) \\
\end{align*}
From (5), (6), and (7), we can determine the total quantity $Q = q_1 + q_2 + q_3$ and the price $P = a - Q$.
\begin{align*}
	Q &= \frac{1}{2} (a - 3c_1 + c_2 + c_3) + \frac{1}{6} (a + 3c_1 - 5c_2 + c_3) + \frac{1}{6} (a + 3c_1 + c_2 - 5c_3) \\
	&= \frac{1}{6} (5a - 3c_1 - c_2 - c_3) \\
	P(Q) &= a - \frac{1}{6} (5a - 3c_1 - c_2 - c_3) \\
	&= \frac{1}{6} (a + 3c_1 + c_2 + c_3)
\end{align*}
The profit for each firm at the subgame-perfect Nash equilibrium is as follows.
\begin{align*}
	\pi_1 &= \frac{1}{6} (a + 3c_1 + c_2 + c_3) \cdot \frac{1}{2} (a - 3c_1 + c_2 + c_3) - c_1 \cdot \frac{1}{2} (a - 3c_1 + c_2 + c_3) \\
	&= \frac{1}{12}(a - 3c_1 + c_2 + c_3)^2\\
	\pi_2 &= \frac{1}{6} (a + 3c_1 + c_2 + c_3) \cdot \frac{1}{6} (a + 3c_1 - 5c_2 + c_3) - c_2 \cdot \frac{1}{6} (a + 3c_1 - 5c_2 + c_3) \\
	&= \frac{1}{36} (a + 3c_1 - 5c_2 + c_3)^2 \\
	\pi_3 &= \frac{1}{6} (a + 3c_1 + c_2 + c_3) \cdot \frac{1}{6} (a + 3c_1 + c_2 - 5c_3) - c_3 \cdot \frac{1}{6} (a + 3c_1 + c_2 - 5c_3) \\
	&= \frac{1}{36} (a + 3c_1 + c_2 - 5c_3)^2 \\
\end{align*}
The total social welfare is the sum of consumer surplus ($CS = (Q^*)^2/2$) and the producer profits. Thus:
\begin{equation}
	TW_a = \frac{(Q^*)^2}{2} + \sum\limits_{i=3}^3 \pi^*_i = \frac{25}{72} (a-c)^2+\frac{1}{12} (a-c)^2+\frac{1}{36} (a-c)^2+\frac{1}{36} (a-c)^2 = \frac{35}{72} (a-c)^2
\end{equation}
\paragraph*{b)}
Since in the initial stage of the game company 1 and 2 choose quantity simulteanously and in the next stage in knowledge of those choice company 3 chooses its own, we first start to analyze the subgame at the end of the: the choice of company 3. Assuming $q_1$ and $q_2$ to be the chosen quantities of the leading companies, the best response of company 3 is (given that all companies are rational and that rationality of companies are common knowledge):
\begin{equation}
	\label{co3profitb}
	\pi_3 = [a - (q_3 + q_{-3})] q_3 - c_3 q_3 ~~ \Rightarrow ~~ \frac{\partial \pi_3}{\partial q_3} = a-2q_3-q_{-3}-c_3=0
\end{equation}
\begin{equation}
	\label{q3bst1}
	q_3(q_1,q_2) = \frac{a-c_3-q_1-q_2}{2}
\end{equation}
This is the strategy of company 3 in the equilibrium of this subgame.
In the first stage, since all companies are rational and that is common knowledge, although company 1 and 2 has no information about previous moves, they can assume rationally that company 3 will have according to (\ref{q3bst1}). Thus their best response functions ($i \in {1,2}$):
\begin{equation}
	\pi_i = [a-(q_1+q_2+\frac{a-c_3-q_1-q_2}{2})]*q_i-c_i*q_i ~~ \Rightarrow ~~ \frac{\partial \pi_i}{\partial q_i} = 0
\end{equation}
\begin{equation}
	\label{q1q2bst2}
	q_1(q_2) = \frac{a-2c_1+c_3-q_2}{2} ~~~~ q_2(q_1) = \frac{a-2c_2+c_3-q_1}{2}
\end{equation}
Since in equilibrium everyone player play their best responses, denoting by an added $*$ the equilibrium quanatities (outcomes), plugging one to the other in (\ref{q1q2bst2}) and rearranging, we get:
\begin{equation}
	q_1^* = \frac{a-4c_1+2c_2+c_3}{3} ~~~~~ q_2^* = \frac{a+2c_1-4c_2+c_+}{3}
\end{equation}
And plugging this back to (\ref{q3bst1}), we get:
\begin{equation}
	q_3^* = \frac{a+2c_1+2c_2-5c_3}{6}	
\end{equation}
Using these, total output, market price and company payoffs are:
\begin{equation}
	Q^* = \frac{5 a-2 c_1-2 c_2-c_3}{6} ~~~~ P^* = \frac{a+2 c_1+2 c_2+c_3}{6}
\end{equation}
\begin{align*}
	\pi_1^* &= \frac{1}{18}(a-4 c_1+2 c_2+c_3)^2 \\
	\pi_2^* &= \frac{1}{18}(a-4 c_1+2 c_2+c_3)^2 \\
	\pi_3^* &= \frac16 (a+2 c_1+2 c_2-5 c_3) \left( \frac16 (a+2 c_1+2 c_2+c_3) -c_3 \right)
\end{align*}
Since the total welfare is the sum of the consumer surplus ($(Q^*)^2/2$) and the 3 profits (assuming $c_1=c_2=c_3=c$):
\begin{equation}
	TW_b = \frac{(Q^*)^2}{2} + \sum\limits_{i=3}^3 \pi^*_i = \frac{25}{72}(a-c)^2+\frac{1}{18}(a-c)^2+\frac{1}{18}(a-c)^2+\frac{1}{36}(a-c)^2=  \frac{35}{72} (a-c)^2
\end{equation}

\paragraph*{c)}
As the structure of the game requires again, first start to analyze the subgame at the end of the game, thus with the decision of company 3 given company 1's and 2's decisions. As shown in (\ref{co3profitb}) and (\ref{q3bst1}), the best response equilibrium strategy of company 3 in the subgame, is:
\begin{equation}
	\label{q3cst1}
	q_3(q_1,q_2) = \frac{a-c_3-q_1-q_2}{2}
\end{equation}
This is the strategy of company 2 in the equilibrium of this subgame.
The next subgame in the 'generalized backward induction' procedure, is the second stage where company 2's decision occurs. Again, since companies are rational and this is common knowledge to everyone, company 2 can rationally assume that company 3 will behave according to (\ref{q3cst1}), thus:
\begin{equation}
	\pi_2 = \left[a-\left(q_1+q_2+\frac{a-c_3-q_1-q_2}{2}\right)\right]q_2-c_2q_2 ~~ \Rightarrow ~~ \frac{\partial \pi_2}{\partial q_2} = 0
\end{equation}
\begin{equation}
	\label{q2cst2}
	q_2(q_1) = \frac{a-2 c_2+c_3-q_1}{2}
\end{equation}
This is the strategy of company 2 in equilibrium of the subgame, where first company 2 chooses quantity and the company 3. Lastly, we'll get to the analysis of the whole game, given the previous sugame equilibrium strategies. Again,since companies are rational and this is common knowledge to everyone, company 1 can rationall assume that company 2 and will behave according to (\ref{q2cst2}) and (\ref{q3cst1}) respectively. Thus:
\begin{equation}
	\pi_1 = \left[ a- \left( q_1+ \frac{a-2 c_2+c_3-q_1}{2}+\frac{a-c_3-q_1-\frac{a-2 c_2+c_3-q_1}{2}}{2} \right) \right]q_1-c_1q_1 ~~ \Rightarrow ~~ \frac{\partial \pi_1}{\partial q_1} = 0
\end{equation}
\begin{equation}
	q_1^* = \frac{a-4c_1+2 c_2+c_3}{2}
\end{equation}
Which is the strategy of company 1 in the equilibrium of the game. Using this, the produced quantities by the companies along the equilibrium strategy profile:
\begin{align*}
	q_2^* &= \frac{a+4 c_1-6 c_2+c_3}{4} \\
	q_3^* &= \frac{a+4 c_1+2 c_2-7 c_3}{8}
\end{align*}
The total output, market price and profits of the companies are:
\begin{equation}
	Q^*= \frac{7 a-4 c_1-2 c_2-c_3}{8} ~~~~ P^* = \frac{a+4 c_1+2 c_2+c_3}{8}
\end{equation}
\begin{align*}
	\pi_1^* &= \frac{1}{16} (a-4 c_1+2 c_2+c_3)^2 \\
	\pi_2^* &= \frac{1}{32} (a+4 c_1-6 c_2+c_3)^2 \\
	\pi_3^* &= \frac{1}{64} (a+4 c_1+2 c_2-7 c_3)^2
\end{align*}
Again, the total welfare is the sum of the consumer surplus ($(Q^*)^2/2$) and the 3 profits (assuming $c_1=c_2=c_3=c$):
\begin{equation}
	TW_c = \frac{(Q^*)^2}{2} + \sum\limits_{i=3}^3 \pi^*_i =\frac{49}{128} (a-c)^2+\frac{1}{16} (a-c)^2+\frac{1}{32} (a-c)^2+\frac{1}{64} (a-c)^2 = \frac{63}{128} (a-c)^2
\end{equation}
\paragraph*{d)}
As we've seen in the solutions, in case of game a), total welfare was ($35/72 (a-c)^2$), in case of game b) it was ($35/72 (a-c)^2$) and for game c), it was ($63/128 (a-c)^2$). Taking the relations between these figures, a social welfare maximizing regulator would prefer the market structure described in point c). The outcome generated by this structure produces even higher consumer welfare than the market structure described in a) and b) ($49/128 > 35/72$). 
\end{document}